% Institute of Computer Science thesis template
% authors: Sven Laur, Liina Kamm
% last change Tõnu Tamme 09.05.2017
\documentclass[12pt]{article}
\usepackage{CormorantGaramond}
\usepackage{amsmath}
\usepackage{amssymb}
\usepackage{amsthm}
\usepackage{array}
\usepackage[english, estonian]{babel}
\usepackage[labelsep = period]{caption}
\usepackage{color}
\usepackage[T1]{fontenc}
\usepackage[a4paper]{geometry}
\usepackage{graphicx}
\usepackage[hidelinks]{hyperref}
\usepackage[all]{hypcap}
\usepackage{inconsolata}
\usepackage[utf8x]{inputenc}
\usepackage{listings}
\usepackage{proof}
\usepackage{tabu}
\usepackage{todonotes}
\usepackage{verbatim}
\usepackage{xcolor}
\usepackage{xspace}
% If you use Estonian make sure that Estonian hyphenation is installed 
% - hypen-estonian or eehyp packages
\addto\captionsestonian{
  \renewcommand\refname{Viidatud kirjandus}
}
\definecolor{hall}{RGB}{128,128,128}
\definecolor{roheline}{RGB}{0,128,0}
\newcommand\peatykk[1]{
  \clearpage
  \section{#1}}
\newcommand\peatykktarn[1]{
  \clearpage
  \section*{#1}
  \addcontentsline{toc}{section}{#1}}
% If you have problems with Estonian keywords in the bibliography
%\usepackage{biblatex}
%\usepackage[backend=biber]{biblatex}
%\usepackage[style=alphabetic]{biblatex}
% plain --> \usepackage[style=numeric]{biblatex}
% abbrv --> \usepackage[style=numeric,firstinits=true]{biblatex}
% unsrt --> \usepackage[style=numeric,sorting=none]{biblatex}
% alpha --> \usepackage[style=alphabetic]{biblatex}
%\DefineBibliographyStrings{estonian}{and={ja}}
%\addbibresource{bachelor-thesis.bib}

% Proper way to create coloured code listings
\lstset{ 
  language=C++,
  basicstyle=\footnotesize,        % the size of the fonts that are used for the code
  numberstyle=\tiny\color{gray}, 
  stepnumber=1,                    % the step between two line-numbers. If it's 1, each line will be numbered
  numbersep=5pt,                   % how far the line-numbers are from the code
  backgroundcolor=\color{white},   % choose the background color. You must add \usepackage{color}
  showspaces=false,                % show spaces adding particular underscores
  showstringspaces=false,          % underline spaces within strings
  showtabs=false,                  % show tabs within strings adding particular underscores
  frame = lines,
  rulecolor=\color{black},       % if not set, the frame-color may be changed on line-breaks within 
                                   % not-black text (e.g. commens (green here))
  tabsize=2,                       % sets default tabsize to 2 spaces
  captionpos=b,                    % sets the caption-position to bottom
  breaklines=true,                 % sets automatic line breaking
  breakatwhitespace=false,         % sets if automatic breaks should only happen at whitespace
  keywordstyle=\color{blue},       % keyword style
  commentstyle=\color{dkgreen},    % comment style
  stringstyle=\color{roheline},       % string literal style
  escapeinside={\%*}{*)},          % if you want to add a comment within your code
  morekeywords={*,game, fun}       % if you want to add more keywords to the set
}
% Macros that make sure that the math mode is set
\newcommand{\opDiv}{\ensuremath{\backslash \mathsf{div}}\xspace} 

% Nice Todo box
\newcommand\markus[1]{\textcolor{roheline}{\textbf{#1}}}
% A way to define theorems and lemmata
\newtheorem{theorem}{Theorem}
\begin{document}
  \thispagestyle{empty}
  \begin{center}
    \large
      TARTU ÜLIKOOL\\
      Arvutiteaduse instituut\\
      Informaatika õppekava\\

    \vspace{25mm}

    \Large
      Liisi Kerik

    \vspace{4mm}

    \huge
      Funktsionaalse programmeerimiskeele liigisüsteem

    \vspace{20mm}

    \Large
      Magistritöö (30 EAP)
  \end{center}

  \vspace{2mm}

  \begin{flushright}
    {
      \setlength{\extrarowheight}{5pt}
      \begin{tabular}{rl} 
        Juhendaja: & Härmel Nestra, PhD
      \end{tabular}}
  \end{flushright}
  \vfill
  \centerline{Tartu 2018}
  \newpage
{
\selectlanguage{estonian}
\noindent\textbf{\large Funktsionaalse programmeerimiskeele liigisüsteem}

\vspace*{1ex}

\noindent\textbf{Lühikokkuvõte:} 

\markus{One or two sentences providing a basic introduction to the field, comprehensible to a scientist in
any discipline.}

\markus{Two to three sentences of
more detailed background, comprehensible to scientists in related disciplines.}

\markus{One sentence clearly stating the general problem being addressed by this particular
study.}

\markus{One sentence summarising the main result (with the words ``here we show´´ or their equivalent).}

\markus{Two or three sentences explaining what
the main result reveals in direct
comparison to what was thought to be the case previously, or how the main result adds to previous knowledge.}

\markus{One or two sentences to put the results into a more general context.}

\markus{Two or three sentences to provide a
broader perspective, readily
comprehensible to a scientist in any
discipline, may be included in the first paragraph
if the editor considers that the accessibility of
the paper is significantly enhanced by their inclusion.}

\vspace*{1ex}

\noindent \textbf{Võtmesõnad:}\\
\markus{List of keywords}

\vspace*{1ex}

\noindent\textbf{CERCS:}\markus{CERCS kood ja nimetus:~\url{https://www.etis.ee/Portal/Classifiers/Details/d3717f7b-bec8-4cd9-8ea4-c89cd56ca46e}}

\vspace*{1ex}}

{
\selectlanguage{english}
\noindent\textbf{\large Type Inference for Fourth Order Logic Formulae}

\vspace*{3ex}

\noindent\textbf{Abstract:}

\noindent
Many interpreting program languages are dynamically typed, such as Visual Basic or Python. As a result, it is easy to write programs that crash due to mismatches of provided and expected data types.  One possible solution to this problem is automatic type derivation during compilation. In this work, we consider study how to detect type errors in the \textsc{Whitespace} language by using fourth order logic formulae as annotations. The main result of this thesis is a new triple-exponential type inference algorithm for the fourth order logic formulae. This is a significant advancement as the question whether there exists such an algorithm was an open question. 
All previous attempts to solve the problem lead lead to logical inconsistencies or required tedious user interaction in terms of interpretative dance. Although the resulting algorithm is slightly inefficient, it can be used to detect obscure programming bugs in the \textsc{Whitespace} language. The latter significantly improves productivity. Our practical experiments showed that productivity is comparable to average Java programmer.   
From a theoretical viewpoint, the result is only a small advancement in rigorous treatment of higher order logic formulae. The results obtained by us do not generalise to formulae with the fifth or higher order. 

\vspace*{1ex}

\noindent \textbf{Keywords:}\\
\markus{List of keywords}
%Layout, formatting, template

\vspace*{1ex}

\noindent\textbf{CERCS:}\markus{CERCS code and name:~\url{https://www.etis.ee/Portal/Classifiers/Details/d3717f7b-bec8-4cd9-8ea4-c89cd56ca46e}}

\vspace*{1ex}
}
\newpage
  \tableofcontents
  \peatykk{Sissejuhatus}
    \markus{What is it in simple terms (title)?}

    \markus{Why should anyone care?}

    \markus{What was my contribution?}

    \markus{What you are doing in each section (a sentence or two per section)}

    Tip: if it's hard for you to start writing, then try to split it to smaller parts, e.g. if the title is ``Type Inference for a Cryptographic Protocol Prover Tool'' then the ``What is it'' can be divided into ``what is type inference'', ``what is cryptographic protocol'' and ``what is the prover tool''. These three can also be split to smaller parts etc.
  \peatykk{Edutamine} 
    \markus{Short description of what this section is about}
    \subsection{Title of Subsection 1}
      Some text...
      \subsubsection{Title of Subsubsection 1}
        Some text...
      \subsubsection{Title of Subsubsection 2}
        Some text...
    \subsection{How to use references} \label{sec:using_ref}
  \peatykk{Keele võimalused}
    \subsection{Süntaks}
      \markus{Need tähed siin on vähemalt pooles ulatuses suvalt pandud...}

      \begin{tabular}{llll}
        Fail                  & $F$ & $::=$ & $I^*D^*K^*F^*$                                                                                                 \\
        Import                & $I$ & $::=$ & {\color{hall}\verb!Load!} $X${\color{hall}\verb!.awf!}                                                         \\
        Nimi                  & $X$ &       &                                                                                                                \\
        Andmetüüp             & $D$ & $::=$ & $S|A|G$                                                                                                        \\
        Struktuur             & $S$ & $::=$ & {\color{hall}\verb!Struct!} $X(${\color{hall}\verb!(!}$V({\color{hall}\verb!,!}V)^*${\color{hall}\verb!)!}$)?$ \\
        Nimi koos tüübiga     & $V$ & $::=$ & $X$ {\color{hall}\verb!:!} $T$                                                                                 \\
        Tüüp                  & $T$ & $::=$ &  \\
        Algebraline andmetüüp & $A$ & $::=$ & {\color{hall}\verb!Algebraic!} $X${\color{hall}\verb!(!}{\color{hall}\verb!)!} \\
        Hargnev andmetüüp     & $G$ & $::=$ &  \\
        Klass                 & $K$ & $::=$ &  \\
        Definitsioon          & $F$ & $::=$ & 
      \end{tabular}

      Keel on tõstutundlik. Erinevalt Haskellist ei ole piiratud, mis nimed peavad algama suur- ja millised väiketähega. Failid, tüübid, tüübimuutujad, konstruktorid, struktuuride väljad, definitsioonide nimed ja lokaalsed muutujad võivad kõik alata kas suur- või väiketähega vastavalt kasutaja soovile.
\begin{comment}
  data Brnch_0 = Brnch_0 Name [Name] Name [(Name, Type_0)] deriving Show
  data Class_0 = Class_0 Name (Name, Kind_0) (Maybe Name) [Method] deriving Show
  data Constraint_0 = Constraint_0 Name Name deriving Show
  data Data_0 = Data_0 Name Data_br_0 deriving Show
  data Data_br_0 = Branching_data_0 Name [Kind_0] [(Name, Kind_0)] [Brnch_0] | Plain_data_0 [(Name, Kind_0)] Data_branch_0
    deriving Show
  data Data_branch_0 = Algebraic_data_0 [Form_0] | Struct_data_0 [(Name, Type_0)]
    deriving Show
  data Def_0 =
    Basic_def_0 Name [(Name, Kind_0)] [Constraint_0] [(Pattern_1, Type_0)] Type_0 Expression_0 |
    Instance_def_0 Location_0 Name Name [Kind_0] [Pattern_1] [Constraint_0] [(Name, ([Pattern_1], Expression_0))]
      deriving Show
  data Expression_0 = Expression_0 Location_0 Expression_branch_0 deriving Show
  data Expression_branch_0 =
    Application_expression_0 Expression_0 Expression_0 |
    Char_expression_0 Char |
    Function_expression_0 Pattern_1 Expression_0 |
    Int_expression_0 Integer |
    Match_expression_0 Expression_0 Matches_0 |
    Name_expression_0 String
      deriving Show
  data Form_0 = Form_0 Name [Type_0] deriving Show
  data Kind_0 = Kind_0 Location_0 Kind_branch_0 deriving (Eq, Show)
  data Kind_branch_0 = Application_kind_0 Kind_0 Kind_0 | Name_kind_0 String deriving (Eq, Show)
  data Match_Algebraic_0 = Match_Algebraic_0 Name [Pattern_1] Expression_0 deriving Show
  data Match_char_0 = Match_char_0 Char Expression_0 deriving Show
  data Match_Int_0 = Match_Int_0 Integer Expression_0 deriving Show
  data Matches_0 =
    Matches_Algebraic_0 [Match_Algebraic_0] (Maybe (Location_0, Expression_0)) |
    Matches_char_0 [Match_char_0] Expression_0 |
    Matches_Int_0 [Match_Int_0] Expression_0
      deriving Show
  data Method = Method Name [(Name, Kind_0)] [Constraint_0] Type_0 deriving Show
  data Pattern_1 = Pattern_1 Location_0 Pattern_0 deriving Show
  data Pattern_0 = Blank_pattern | Name_pattern String deriving Show
  newtype Parser t = Parser {parser :: State -> Either Location_0 (t, State)}
  data State = State Tokens Location_0 deriving Show
  data Type_0 = Type_0 Location_0 Type_branch_0 deriving Show
  data Type_branch_0 =
    Application_type_0 Type_0 Type_0 | Name_type_0 String [Kind_0] deriving Show
\end{comment}
    \subsection{Näited}
  \peatykk{Edasine töö}
    \subsection{Liik kui kategooria}
    \subsection{Klass kui alamliik}
    \subsection{Kasutusmugavus}
    \subsection{Süntaksi kasutajapoolsete täienduste võimalus}
\paragraph{Cross-references to figures, tables and other document elements.}
LaTeX  internally numbers all kind of objects that have sequence numbers:
\begin{itemize}
\item chapters, sections, subsections;
\item figures, tables, algorithms;
\item equations, equation arrays.
\end{itemize}
To reference them automatically, you have to generate a label using \texttt{$\backslash$label\{some-name\}} just after the object that has the number inside. Usually, labels of different objects are split into different namespaces by adding dedicated prefix, such as \texttt{sec:}, \texttt{fig:}. To use the corresponding reference, you must use command \texttt{$\backslash$ref} or \texttt{$\backslash$eqref}. For instance, we can reference this subsection by calling Section~\ref{sec:using_ref}. Note that there should be a nonbreakable space \texttt{\~} between the name of the object and the reference so that they would not appear on different lines (does not work in Estonian).          

\paragraph{Citations.}
Usually, you also want to reference articles, webpages, tools or programs or books. For that you should use citations and references. The system is similar to the cross-referencing system in LaTeX. For each reference you must assign a unique label. Again, there are many naming schemes for labels. However, as you have a short document anything works. To reference to a particular source you must use \texttt{$\backslash$cite\{label\}} or \texttt{$\backslash$cite[page]\{label\}}. 

References themselves can be part of a LaTeX source file. For that you need to define a bibliography section. However, this approach is really uncommon. It is much more easier to use BibTeX to synthesise the right reference form for you. For that you must use two commands in the LaTeX source
\begin{itemize}
\item $\backslash$bibliographystyle\{alpha\} or $\backslash$bibliographystyle\{plain\}
\item $\backslash$bibliography\{file-name\}
\end{itemize}
The first command determines whether the references are numbered by letter-number combinations or by cryptic numbers. It is more common to use \texttt{alpha} style. The second command determines the file containing the bibliographic entries. The file should end with \texttt{bib} extension. Each reference there is in specific form. The simplest way to avoid all technicalities is to use graphical frontend  Jabref (\url{http://jabref.sourceforge.net/}) to manage references. Another alternative is to use DBLP database of references and copy BibTeX entries directly form there.   
    
   
The following paragraph shows how references can be used. Game-based proving is a way to analyse security of a cryptographic protocol~\cite{GameB_1, GameB_2}. There are automatic provers, such as {CertiCrypt\-}~\cite{certicrypt} and ProVerif~\cite{proVerif}.

\newpage
\begin{comment}
\section{How to add figures and pictures to your thesis}


Here are a few examples of how to add figures or pictures to your thesis (see Figures~\ref{fig:fnCompModel}, \ref{fig:game-based_proofs}, \ref{fig:proveit_screenshot}).

Rule: All the figures, tables and extras in the thesis have to be referred to somewhere in the text.


\begin{figure} [ht] %try to place the figure here (next option top of the page) 
\begin{center}
\includegraphics[width=0.8\textwidth]{computational_model_function}
\caption{The title of the Figure.}
\label{fig:fnCompModel}
\end{center}
\end{figure}



\begin{figure} [!ht] %if [h] doesn't work, we can force with !
\begin{center}
\includegraphics[width=\textwidth]{game-based_proofs}
\caption{Refer if the figure is not yours~\cite{kamm12}.}
\label{fig:game-based_proofs}
\end{center}
\end{figure}


\begin{figure} [p]
\begin{center}
\includegraphics[width=\textwidth]{proveit_screenshot}
\caption{Screenshot of \proveit.}
\label{fig:proveit_screenshot}
\end{center}
\end{figure}

Tip: If you add a screenshot then labeling the parts might help make the text more understandable (panel C vs bottom left part), e.g.


\begin{figure} [htbp]
\begin{tabular}{c c}
%
\begin{minipage}{0.45\textwidth}
\includegraphics[width=\textwidth]{LCA_2_solutions}
\end{minipage}
%
&
\begin{minipage}{0.55\textwidth}
\centering
\begin{tabular}{ l | l |}
  Node & Decendants \\ \hline
  1 & 2, 3, 4 \\ \hline
  2 & 3, 4 \\ \hline
  3 & \\ \hline
  4 & \\ \hline
  5 & 3, 4, 6, 7 \\ \hline
  6 & 4 \\ \hline
  7 & 3 \\  \hline
  8 & 3, 4, 5, 6, 7\\ \hline
  9 & 3, 4, 5, 6, 7\\ \hline
\end{tabular}
\end{minipage}
\end{tabular}
%
\caption{Example how to put two figures parallel to each other.}
\label{fig:LCA_2_solutions}
\end{figure}


Example: A screenshot of \proveit can be seen on Figure~\ref{fig:proveit_screenshot}. The user first enters the pseudocode of the initial game in panel B. \proveit also keeps track of all the previous games showing the progress on a graph seen in panel A.

There are two figures side by side on Figure~\ref{fig:LCA_2_solutions}.
\end{comment}
  \peatykk{Other Ways to Represent Data}

\subsection{Tables}

\begin{table}[h]
\centering
\caption{Statements in the proveit language.}
\begin{tabular}{| l | l |}
  \hline
  \bf{Statement} & \bf{Typeset Example} \\
  \hline
  assignment & $a := 5 + b$ \\
  \hline
  uniform choice & $m <- M$ \\
  \hline
  function signature & $f : K \times M -> L$\\
  \hline
\end{tabular}
\label{tab:statements}
\end{table}


\subsection{Lists}

Numbered list example:
\begin{enumerate}
  \item item one; 
  \item item two;
  \item item three.
\end{enumerate} 

\subsection{Math mode}
Example:
\begin{equation}
a + b = c + d
\end{equation}
Aligning:
\begin{align*}
  a &= 5 \\
  b + c &= a \\
  a -2*3 &= 5/4
\end{align*}
    \subsection{Frame Around Information}
Tip: We can use minipage to create a frame around some important information.
\begin{figure} [h]
\frame{
\begin{minipage}{\textwidth}
\begin{enumerate}
  \item integer division ($\opDiv$)
  \item remainder ($\%$)
\end{enumerate}
\end{minipage}
}
\caption{Arithmetic operations in proveit revisited.}
\label{fig:aritmOp_revisit}
\end{figure}
  \peatykk{Kokkuvõte}

\markus{what did you do?}

\markus{What are the results?}

\markus{future work?}

\newpage
\bibliographystyle{plain}
\bibliography{bachelor-thesis}

\addcontentsline{toc}{section}{\refname}

% Use Biblatex if you have problems with Estonian keywords
%\printbibliography %biblatex
  \peatykktarn{Litsents}
    \subsection*{Lihtlitsents lõputöö reprodutseerimiseks ja lõputöö üldsusele kättesaadavaks tegemiseks}
      Mina, \textbf{Liisi Kerik},

      \begin{enumerate}
        \item
          annan Tartu Ülikoolile tasuta loa (lihtlitsentsi) enda loodud teose

          \textbf{Funktsionaalse programmeerimiskeele liigisüsteem}

          mille juhendaja on Härmel Nestra

          \begin{enumerate}
            \item[1.1]
              reprodutseerimiseks säilitamise ja üldsusele kättesaadavaks tegemise eesmärgil, sealhulgas digitaalarhiivi DSpace-is lisamise eesmärgil kuni autoriõiguse kehtivuse tähtaja lõppemiseni;
            \item[1.2]
              üldsusele kättesaadavaks tegemiseks Tartu Ülikooli veebikeskkonna kaudu, sealhulgas digitaalarhiivi DSpace´i kaudu kuni autoriõiguse kehtivuse tähtaja lõppemiseni.
          \end{enumerate}
        \item
          olen teadlik, et punktis 1 nimetatud õigused jäävad alles ka autorile.
        \item
          kinnitan, et lihtlitsentsi andmisega ei rikuta teiste isikute intellektuaalomandi ega isikuandmete kaitse seadusest tulenevaid õigusi. 
      \end{enumerate}

      \noindent Tartus, 21.05.2018
\end{document}